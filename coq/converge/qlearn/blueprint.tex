\documentclass{article}
\usepackage[utf8]{inputenc}

\usepackage{graphicx}
\usepackage{float}
% \usepackage{ebgaramond-maths}
\usepackage{fullpage}	
\usepackage{amsfonts}
\usepackage{color}
\usepackage{lipsum}
\usepackage{wrapfig}
\usepackage{amsmath, amssymb, amsfonts, amscd, amsthm, mathrsfs}
\usepackage{tabularx}
\usepackage{framed}
\usepackage{longtable,tabu}
\usepackage{hyperref}
\usepackage{pifont}
\hypersetup{
    colorlinks=true,
    linkcolor=blue,
    filecolor=magenta,      
    urlcolor=cyan,
}

% \usepackage{pgfplots}
% \pgfplotsset{compat=1.15}
% \usepackage{mathrsfs}
% \usetikzlibrary{arrows}
% \pagestyle{empty}


% \theoremstyle{plain}
% \usepackage{ntheorem}

\title{Q-learning convergence : A Formal Blueprint}
\author{Koundinya Vajjha}

\def\R{\mathbb{R}}
\def\dmin{D_{\min}}
\def\Cinf{C^\infty(M)}
\def\star{\mathfrak{h}^{\ast}}
\def\E{\mathbb{E}}
\def\act{\mathcal{A}}
\def\rew{\mathcal{R}}

\newcommand{\figurehere}[2]{\smallskip\begin{center} 
\includegraphics[width=#1\textwidth]{#2}\end{center}\smallskip}

\newcommand{\image}[1]{\smallskip\begin{center} #1 \end{center}\smallskip}
\newcommand{\red}[1]{\textcolor{red}{#1}}
% environments for theorems, lemmas, etc.
\newtheorem{lem}{Lemma}[section]
\newtheorem{theorem}{Theorem}
\newtheorem{thm}{Theorem}
\newtheorem{ex}[lem]{Example}
\newtheorem{lemma}[theorem]{Lemma}
\newtheorem*{conjecture}{Conjecture}
\newtheorem*{remark}{Remark:}
% \newtheorem{lem}[theorem]{Lemma}
\newtheorem{corollary}[theorem]{Corollary}
\newtheorem{cor}[theorem]{Corollary}
\newtheorem{proposition}[theorem]{Proposition}
\newtheorem{prop}[theorem]{Proposition}
\newtheorem{defn}[theorem]{Definition}
\newcommand{\code}[1]{\texttt{#1}}
\DeclareMathOperator*{\argmax}{argmax}
\DeclareMathOperator*{\argmin}{argmin}
\newcommand{\fleur}{\ding{95}}
\newcommand{\HTMLBase}{https://certrl.github.io/CertRLanon}
\newcommand{\coqHTMLBase}{\HTMLBase}
\newcommand{\coqBaseModule}{CertRL.}
\newcommand{\coqtop}{\text{\href{https://github.com/CertRL/CertRLanon}{\fleur}}}
\newcommand{\coqdef}[2]{\text{\href{\coqHTMLBase/\coqBaseModule#1.html\##2}{\fleur}}}



\begin{document}

\maketitle

\section{Introduction}
This will be a formal blueprint of the convergence proof for Q-learning, following the original 1992 proof of Watkins-Dayan. Formal equivalents, whenever available, shall be linked to by a \coqtop.

In a nutshell, the proof proceeds by constructing an auxiliary Markov Decision Process, called the \textit{action-replay} process (ARP) associated to any given MDP. The ARP is constructed so that it's optimal $Q$-values (defined below) approximate the optimal $Q$-value of the parent MDP. 
A stochastic approximation result (the Robbins-Monro procedure) finally shows that the approximations indeed converge to the optimal value.

We start out with an MDP $M$ \coqdef{converge.mdp}{MDP} which has 
\begin{itemize}
    \item A nonempty, finite state space $X$.
    \item An nonempty, finite action space $\act$ (fibered over each state $x:X$). 
    \item A stochastic transition structure $T$ which assigns a probability measure over the state space given a state $x$ and an action $a : \act(x)$ available at that state. 
    \item A reward function $r$. \coqdef{converge.mdp}{reward}
\end{itemize}
The expected reward after one transition from state $x$ after taking an action $a$ is called the \textit{step expected reward} \coqdef{converge.mdp}{step_expected_reward} and is denoted $\rew_x(a)$

A \textit{decision rule}\coqdef{converge.mdp}{dec_rule} is a map from the state space to the action space. Transitions happen after actions are taking in a particular state space following a decision rule.
Denote by $T_\pi (x) := T(x,\pi(x))$ which is the probability measure on $X$ which results after one transition following the decision rule $\pi$ at state $x$. 

If we use the same decision rule at each time step $n$, such policies are called \textit{stationary}. Consider only stationary policies from now. 

The long-term value \coqdef{converge.mdp}{ltv} of a policy $\pi$ is denoted by $V^{\pi}$ and is shown to be equal to
\[ 
V^{\pi}(x) = \rew_x(\pi(x)) + \gamma \E_{T_{\pi}(x)} [V^{\pi}]
\]
where $0 \le \gamma \le 1$ is a discount factor and the expectation is with respect to the probability measure $T_{\pi}(x)$.

\begin{defn}[$Q$-value]
Given an MDP $M$ and a policy $\pi$, we define the $Q$-value as:
\[ 
Q^{\pi}(x,a) := \rew_x(a) + \gamma \E_{T(x,a)}V^{\pi} 
\]
\end{defn}

The value and policy iteration algorithms give us an \textit{optimal policy} denoted $\pi^*$ which maximizes the long-term value. This value is called the optimal value function denoted $V^{\pi^*} := \max_{\pi}(V^\pi)$. \coqdef{converge.mdp}{max_ltv}

Define $Q^*(x,a) := Q^{\pi^*}(x,a)$. We can show that 
\[ 
V^*(x) := V^{\pi^*}(x) = \max_{a}\left( \rew_x(a) + \gamma\E_{T(x,a)}V^{\pi^*}\right) = \max_{a}Q^{\pi^*}(x,a) = \max_{a}Q^*(x,a)
\]
which says that the optimal value function is the fixed point of the optimal Bellman operator. \coqdef{converge.mdp}{max_ltv_eq_fixpt}

So if an agent can learn the $Q^*$ value, then we can learn $\pi^*$ by simply taking $\pi^*(x) = \argmax_{a} Q^*(x,a)$.

The evolution of an MDP proceeds in stages. At the $n$-th stage, the agent observes state $x_n$, takes an action $a_n$, observes a subsequent stage $y_n$, receives an immediate payoff $r_n$ and adjusts it's $Q_{n-1}$ value (defined below) using a learning factor $\alpha_n$. 

\begin{defn}[$Q_n$-value]
Assume that $Q_0(x,a)$ is a known function for all $x : X$ and $a : \act(x)$. 
For $0 \le \alpha_n < 1$ define inductively the following sequence 
\[Q_n(x,a) :=
\begin{cases}
(1 - \alpha_n)Q_{n-1}(x,a) + \alpha_n\left[ r_n + \gamma V_{n-1}(y_n) \right] \mathrm{ if } \ x=x_n \ \mathrm{ and } \ a=a_n \\
Q_{n-1}(x,a) \quad \mathrm{ otherwise}
\end{cases}
\]
where $r_n = r(x_n,a_n)$ and $V_{n-1}(y) = \max_a Q_{n-1}(y,a)$.
\end{defn}
In practice, the $Q_0$ value is set to zero identically. 
$Q$-learning states that under certain regularity conditions on the learning rate, the sequence of $Q_n$ values converges to $Q^*$ with probability 1. 

\begin{theorem} [$Q$-learning convergence]
Fix a state $x:X$ and action $a : \act(x)$. 
If we have, for all $x,a$ that
\[
\sum_{i}\alpha_{n^i(x,a)} = \infty \ , \ \sum_{i}\alpha_{n^i(x,a)}^2 < \infty \ 
\]
then we have $Q_n(x,a) \rightarrow Q^*(x,a)$ with probability 1. 
\end{theorem}

All random variables in sight and all probability measures are discrete. 

\section{Action-Replay Process Lemmas}
To prove this theorem, we set up an auxilary MDP called the Action-Replay Process (ARP). 
\subsection{Definition of the ARP.}
\begin{defn}[$i$-th occurence counter]
Given an MDP, let $n^i(x,a)$ denote the time when action $a$ is tried at state $x$ for the $i$-th time. 
\end{defn}

\begin{defn}[Action-Replay Process]
Given an MDP $M = (X,A,T,r)$, we define the action-replay process ARP as having 
\begin{itemize}
    \item State space $(\mathbb{N}\times X) \cup \{\ast\}$. Where $\ast$ is a special absorbing state.
    \item Action space $A$ is the same as the action space of $M$.
    \item Transition structure TODO. 
\end{itemize}
\end{defn}

\subsection{Optimal $Q$ values of the ARP.}

\section{Stochastic Convergence Lemmas}
\subsection{Convergence in $L^2 \Rightarrow$ Convergence with probability 1.}
\textbf{Note}: \red{This section may be optional. Maybe we should think about stating our convergence results just in $L^2$, since convergence in probability doesn't really add much?}

\begin{lemma}[Markov Inequality]
Let $X$ be a discrete random variable which is nonnegative. Then for any positve real number $a$, we have 
\[ 
P(X \ge a) \le \frac{\E X}{a}
\]
\end{lemma}
\begin{proof}
We have $\E X= \sum_{x} x P(X = x)$ \coqdef{converge.cond_expt}{expt_value_eq_class} and we now write:
\begin{align}
    \E X &= \sum_{x \ge a} x P(X = x) + \sum_{x < a} x P(X = x) \\
        &\ge \sum_{x \ge a} x P(X=x) + 0 \\
        &\ge a \sum_{x \ge a} P(X = x) \\
        &= a P(X \ge a)
\end{align}
where we used the nonnegativity of $X$ in the second step. 
\end{proof}

Using the Markov inequality, we can prove that convergence in $L^2$ implies convergence in probability. 

\begin{defn}[Convergence in $L^2$]
A sequence of random variables $X_n$ is said to converge in $L^2$ to a random variable $X$ if \[ 
\lim_{n \to \infty} \E |X_n - X|^2 = 0.
\]
\end{defn}

\begin{defn}[Convergence with probability 1]
A sequence of random variables $X_n$ is said to converge with probability 1 to a random variable $X$ if
\[
    \lim_{n \to \infty} P \left( | X_n - X | \ge \epsilon \right) = 0.
\]
\end{defn}
\begin{theorem}
Convergence in $L^2$ implies convergence in probability. 
\end{theorem}
\begin{proof}
For any $\epsilon > 0$, we have 
\begin{align}
    P(|X_n - X|) \ge \epsilon) &= P(|X_n - X|^2 \ge \epsilon^2) \\
    &\le \frac{\E |X_n - X|^2}{\epsilon^2}
\end{align}
The RHS now goes to 0 since $X_n \to X$ in $L^2$. 
\end{proof}

\subsection{Auxilary Lemmas}

Here are a list of auxilary lemmas which we will use in the next section.

\begin{lemma}\label{lem:tailszero}
    Given $\{a_n\}$ was a sequence of real numbers such that $\sum_{n=0}^\infty a_n<\infty$ then we have that the sequence of tails
    $\{\sum_{n=m}^\infty a_n\} \rightarrow 0$.
    \end{lemma}
    
    \begin{lemma}\label{lem:partprodbdd}
    If $\{F_n\}$ is a sequence of positive real numbers such that $\prod_{n=1}^{\infty} F_n =0$, then all partial products $\prod_{n=r}^s F_n^2$ are uniformly bounded by a finite number A. 
    \end{lemma}
    
    \begin{lemma}\label{lem:maxzero}
    If $\{F_n\}$ is a sequence of positive real numbers such that $\prod_{n=1}^{\infty} F_n =0$, and if $m$ is a fixed natural number then we have
    \[ \max_{1 \le k < m}\prod_{j=k+1}^n F_j^2  \rightarrow 0 \]
    as $n \rightarrow \infty$
    \end{lemma}
    \begin{lemma}\label{thm:fprod}
        If we have $\sum_{n=0}^\infty \alpha_n = \infty$ where $0 \le \alpha_n < 1$ and if $F_n = 1 - \alpha_n$, then $\prod_{n=0}^\infty F_n = 0$. 
        \end{lemma}
        \begin{proof}
            We have $F_n = 1 - \alpha_n \le e^{-\alpha_n}$ (this result is proven in \texttt {RealAdd.v}) and so,
            \[ 
            0 \le \prod_{n=0}^k (1 - \alpha_k) \le e^{- \sum_{n=0}^k \alpha_k}
            \]
            By the hypothesis, the RHS goes to zero as $k \to \infty$. So, by the sandwich theorem we have $\prod_{n=0}^\infty F_n = 0$. 
        \end{proof}
        
    \begin{theorem}\label{thm:recle}
        Let $\{F_n\}$ be a sequence of positive real numbers such that $\prod_{n=0}^\infty F_n =0$. Assume we have a sequence of real numbers $\{V_n\}$ which satisfy for every $n$:
        \[ 
            V_{n+1}^2 \le F_n^2 V_n^2 + \sigma_n^2
        \]
        Where $\{\sigma_n\}$ satisfy $\sum_{n=0}^\infty \sigma_n^2 < \infty$. Then we have that $V_n^2 \rightarrow 0$.
        \begin{proof}
        Iterating the inequality we have:
        \begin{align}
            V_{n+1}^2 &\le  \sigma_n^2 + \sigma_{n-1}^2 F_n^2 + \dots + \sigma_{m}^2 F_{m+1}^2\dots F_n^2 +\dots+ \sigma_{1}^2 F_{2}^2 F_3^2\dots F_n^2 + V_1^2 F_2^2\dots F_n^2 \\
            &=  \sum_{p=1}^n \sigma_p^2 \prod_{i=p+1}^n F_i^2 + V_1^2 F_2^2\dots F_n^2
        \end{align}
        
        Now to prove that $V_{n+1}$ can be made arbitrarily small, we observe that for a fixed but arbitrary $m$ with $1\le m \le n$:
        \begin{align}
            V_{n+1}^2 \le \sum_{j=m}^n \sigma_j^2 \max_{m \le k \le n} \prod_{j=k+1}^n F_j^2 + \left(V_1^2 + \sum_{j=1}^{m-1} \sigma_j^2\right)\max_{1 \le k < m}\prod_{j=k+1}^n F_j^2
        \end{align}
        
        By the hypothesis and lemma \ref{lem:partprodbdd} we have that all partial products $\prod_{n=r}^s F_n^2$ are uniformly bounded by a finite number $A$.
        Given an $\epsilon >0$ choose $m$ so large that $A (\sum_{j=m}^\infty \sigma_j^2) < \epsilon/2$. 
        From this and equation (3) we have: 
        \begin{align}
             V_{n+1}^2 &< A \sum_{j=m}^\infty \sigma_j^2  + \left(V_1^2 + \sum_{j=1}^{m-1} \sigma_j^2\right)\max_{1 \le k < m}\prod_{j=k+1}^n F_j^2 \\
             &< \epsilon/2 + \left(V_1^2 + \sum_{j=1}^\infty \sigma_j^2\right)\max_{1 \le k < m}\prod_{j=k+1}^n F_j^2 \\
        \end{align}
        Now using lemma \ref{lem:maxzero} we get that the max term can be made arbitrarily small as $n \rightarrow \infty$. Hence for $n$ sufficiently large, we get that $V_{n+1}^2 \le \epsilon$.
        \end{proof}
        
        
        \end{theorem}
        
        
        % \section{Proof of (2.5) + (8.1) $\Rightarrow$ (8.3) in Dvoretzky's paper.}
        % Let $\{\alpha_n\},\{\beta_n\},\{\gamma_n\}$ be sequences of nonnegative real numbers and let $\theta$ be a real number, such that
        % \begin{itemize}
        %     \item $\mathrm{lim}_{n\rightarrow \infty}{\alpha_n} = 0$
        %     \item $\sum_{n=1}^{\infty}{\beta_n} < \infty$
        %     \item $\sum_{n=1}^{\infty} \gamma_n = \infty$
        % \end{itemize}
        % Let $T$ be a function satisfying the following assumption:
        % \[ 
        % |T_n(r_1,\dots,r_n) - \theta| \le F_n | r_n - \theta |
        % \]
        % Define $\{X_n\}$ as
        % \[
        % X_{n+1}(\omega) = T_n(X_1(\omega),\dots,X_n(\omega)) + Y_n(\omega) \qquad n \ge 1 
        % \]
        % assume also that
        % \begin{itemize}
        %     \item $\mathbb{E}[X_1^2] < \infty$,
        %     \item $\sum_{n=1}^{\infty}\E[Y_n^2] < \infty$,
        %     \item $\E[Y_n | x_1,\dots,x_n] = 0$ for all $n$ with probability 1,
        % \end{itemize}
\subsection{Dvoretzky's convergence lemma}

We now prove the weak Dvoretzky convergence theorem:

\begin{theorem}[Dvoretzky's weak theorem]
Let $T$ be a function satisfying the following assumption:
\begin{equation}\label{eq:dvoret}
|T_n(r_1,\dots,r_n) - \theta| \le F_n | r_n - \theta |    
\end{equation}

For a positive sequence of real numbers $\{F_n\}$ such that $\prod_{n=0}^\infty F_n =0$.
Define $\{X_n\}$ as
\[
X_{n+1}(\omega) = T_n(X_1(\omega),\dots,X_n(\omega)) + Z_n(\omega) \qquad n \ge 1 
\]
assume also that
\begin{itemize}
    \item $\mathbb{E}[X_1^2] < \infty$,
    \item $\sum_{n=1}^{\infty}\E[Z_n^2] = \sum_{n=1}^{\infty}\sigma_n^2 < \infty$,
    \item $\E[Z_n | X_1,\dots,X_n] = 0$ for all $n$ with probability 1,
\end{itemize}

Now if we set $V_n^2 = \E\left[ (X_n - \theta)^2\right]$ and $\sigma_n^2 = \E\left[ Z_n^2 \right]$, then we have:
\begin{enumerate}
    \item $V_{n+1}^2 \le F_n^2 V_n^2 + \sigma_n^2$ and
    \item $V_n^2 = \E\left[ (X_n - \theta)^2\right] \to 0$. That is, $X_n \to \theta$ in $L^2$ and so $X_n \to 0$ with probability 1. 
\end{enumerate} 

\end{theorem}
\begin{proof}
    Let us prove (1). We have:
    \begin{align}
    V_{n+1}^2 &= \E\left[(X_{n+1} - \theta)^2\right] \\
              &= \E\left[ (T_n + Y_n - \theta)^2 \right] \\
              &= \E\left[ (T_n - \theta)^2 + Y_n^2 + 2Y_n\left[T_n - \theta\right] \right] \\
              &= \E\left[(T_n - \theta)^2\right] + \sigma_n^2 + 2 \E[(T_n - \theta)Y_n] \\
              &\le F_n^2 V_n^2 + \sigma_n^2 + 2\E\left[\E\left[(T_n - \theta)Y_n | X_1,\dots,X_n]\right]\right] \\
              &=  F_n^2 V_n^2 + \sigma_n^2 + 2\E\left[(T_n - \theta)\E\left[Y_n | X_1,\dots,X_n]\right]\right] \\
              &= F_n^2 V_n^2 + \sigma_n^2
    \end{align}
    since the conditional expectation of $Y_n$ w.r.t the $X_k$'s is 0. In steps (10) and (11) we use the tower law of conditional expectation and the ``take-out-what-is-known'' property of conditional expectations.
    
    We can now apply Lemma \ref{thm:recle} to conclude that $V_n^2 \rightarrow 0$, which means that $X_n \rightarrow \theta$ in $L^2$. By the Markov inequality, one can also conclude that $X_n \rightarrow \theta$ with probability 1.
\end{proof}

\section{Applications of the weak Dvoretzky convergence theorem.}
\subsection{Robbins-Monro procedure}
We outline a proof of convergence of the classical Robbins-Monro procedure.
\begin{lemma}\label{lem:qlearn-ess}
Let $\{X_n\}$ be discrete random variables updated as
\begin{equation}\label{eq:rob-monro}
X_{n+1} = X_n + \alpha_n (Y_n - X_n)    
\end{equation}
where $ 0 \le \alpha_n < 1$ are real numbers such that $\sum_n \alpha_n = \infty$ and $\sum_n \alpha_n^2 < \infty$ and $Y_n$ are bounded random variables with mean $\theta$. Then we have that $X_n \to \theta$ in $L^2$.
\end{lemma}
\begin{proof}
% \begin{remark}
% We note that
% \begin{itemize}
%     \item Convergence is also in probability if we use the lemmas in the previous section.
%     \item The above recurrence scheme (\ref{eq:rob-monro}) is the Robbins-Monro scheme with $f(x) = x$. (See Dvoretzky's paper on page 50).
% \end{itemize} 
% \end{remark}

First off, note that in Lemma \ref{lem:qlearn-ess}, we can assume, without loss of generality, that $\theta = 0$. 
This is because we can rewrite (\ref{eq:rob-monro}) as 
\[ 
    (X_{n+1} - \theta)= (X_n - \theta) + \alpha_n ((Y_n-\theta) - (X_n-\theta))    
\] 
So, to prove Lemma \ref{lem:qlearn-ess}, we apply the Dvoretzky convergence lemma with $Z_n = \alpha_n Y_n$ and $T_n = (1 - \alpha_n)X_n$, which implies that $F_n = (1 - \alpha_n)$. 

Note that $$\sum_n\E[Z_n^2] = \sum_n \alpha_n^2 \E[Y_n^2] \le D \sum_n \alpha_n^2 < \infty$$ since we assume that $Y_n$ are bounded. The missing ingredient is supplied by Lemma \ref{thm:fprod}.
\end{proof}

\subsection{Proof of Jaakkola Lemma 1}
In Jaakkola's paper, Lemma 1 says the following:
\begin{theorem}
A random process 
\[ 
w_{n+1}(x) = (1 - \alpha_n(x))w_{n}(x) + \beta_n(x)r_n(x) \qquad n \ge 1
\]
converges to zero in $L^2$ (hence with probability 1) if the following conditions are satisfied:
\begin{itemize}
    \item $\sum_{n} \alpha_n(x) = \infty$ , $\sum_n \alpha_n(x)^2 < \infty$, $\sum_n \beta_n^2(x) < \infty$ uniformly over $x$ with probability 1. 
    \item $\E\left[ r_n(x) | P_n , \beta_n\right] = 0$ and $\E\left[ r_n^2(x) | P_n, \beta_n \right] \le C$ with probability 1, where
    \[ P_n = \{w_n,w_{n-1}\dots,r_{n-1},r_{n-2},\dots,\alpha_n,\alpha_{n-1},\dots,\beta_n,\beta_{n-1}\dots \}\]
    is the information available at the current moment. $\alpha_n(x)$ and $\beta_n(x)$ are assumed to be non-negative and mutually independent w.r.t $P_n$.
\end{itemize}
\end{theorem}
\begin{proof}
We shall sketch a proof of this in the case where 
\begin{enumerate}
    \item $\alpha_n(x)$ and $\beta_n(x)$ are constant functions for $n \ge 1$.
    \item  $0 \le \alpha_n < 1$.
    \item When $\E\left[ w_1^2 \right] < \infty$
\end{enumerate}
Although the results should all pull through with a bit of modification even without the first assumption. 

We apply the weak Dvoretzky theorem. The following list shows that all the hypotheses of that theorem are satisfied in our case with the assumptions.  
\begin{itemize}
    \item In our case, set $T_n = (1 - \alpha_n)w_n$ as before, which implies $F_n = 1 - \alpha_n$. (\ref{eq:dvoret}) is trivially satisfied. 
    \item We have $Z_n = \beta_n r_n$. So we have $\E\left[Z_n | w_n,w_{n-1},\dots,w_1\right]= \E\left[r_n  \beta_n| w_n,w_{n-1},\dots,w_1\right] = 0$ from the hypothesis and since $\beta_n$ was assumed constant.
    \item We also have \[ \sum_n\E[Z_n^2] = \sum_n \beta_n^2 \E[r_n^2] \le C \sum_n \beta_n^2 < \infty \]
    \item Note that $\sum_n \alpha_n = \infty$ and Lemma \ref{thm:fprod} imply that $\prod_{n=0}^{\infty} F_n = 0$. 
    
\end{itemize}


\end{proof}

\end{document}

